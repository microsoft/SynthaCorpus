\documentclass{article}
\usepackage[margin=1in]{geometry}
\usepackage{url}
\usepackage{color}
\usepackage[dvipsnames]{xcolor}
\usepackage{listings}

\newcommand{\red}[1]{{\color{red}{\emph{#1}}}}
\newcommand{\projectName}{\texttt{SynthaCorpus}}
\newcommand{\projectRoot}{\texttt{.../SynthaCorpus}}


\title{Developer Overview of \projectName}
\author{Microsoft}

\begin{document}
\maketitle{}

The \projectName~project is written in C11 and perl 5.  Development
has been carried out under cygwin on Windows 10, using gcc version
5.4.0 and perl 5.22.  The C executables can also be built under Visual
Studio 2015 and solution/project files are included in the
distribution to facilitate this.  Successful building and basic
operation under Bash on Windows-10 has been confirmed though more
thorough testing is needed. I don't expect major problems porting to
other environments.  However, because large data sizes lead to heavy
memory demands, there is a strong assumption that the executables will
be built for and run on 64-bit architecture.


In the following it is assumed that the root of the project directory
structure is at \projectRoot.  All references to directory paths are relative to
that.

\section{Licenses, Authorship and Copyright}
The bulk of the code in this project was written by Microsoft.

Third-party open-source code and content is also included as follows:

\begin{description}
  \item [Project Gutenberg books] - the directory
    \texttt{ProjectGutenberg} contains 49 popular books in text
    format, downloaded from \url{http://www.gutenberg.org/}.  These
    are books which are no longer subject to copyright.  They
    are provided as a small test set to illustrate and verify the
    operation of project \projectName.
    \item [Fowler Noll Vo hash function] - this is a hashing function
      which we have consistently found to be fast and effective.  The code
      is at \texttt{src/imported/Fowler-Noll-Vo-hash}.  Both
      \texttt{fnv.c} and \texttt{fnv.h} in that directory include a
      licence allowing free distribution.  See \url{http://www.isthe.com/chongo/tech/comp/fnv/}.
      \item [Tiny Mersenne Twister random number generator] - this is
        a fast random number generator with very good properties.  We
        chose to use the tiny version (still with excellent
        properties), to allow the possibility of corpus generation
          being built into an indexer being studied.  The code is at
          \texttt{src/imported/TinyMT\_cutdown}.  The file
          \texttt{readme.html} contains the license.
          See \url{http://www.math.sci.hiroshima-u.ac.jp/~m-mat/MT/TINYMT/index.html}
      \end{description}
    


\section{Project dependencies}
Successful building and operation of the \projectName~project requires
installation of the following packages.  Version numbers shown are the
ones used.  Other versions may be suitable.

\begin{itemize}
\item \texttt{gcc 5.4.0} and \texttt{make 4.2.1} OR VisualStudio 2015
\item \texttt{perl 5.22}
  \item \texttt{gnuplot 5.0} -- for displaying extracted corpus properties
  \item \texttt{pdflatex 2.9.4902} -- for compiling documentation (if needed)
    \item \texttt{acroread 2017 version} OR other PDF viewer -- for viewing documentation
      and plots.
\end{itemize}



\section{Building executables and documents}
\subsection{With gcc}

\begin{verbatim}
cd src
make cleanest
make
\end{verbatim}

builds all the executables.  One warning is emitted, complaining about a function which is not
referenced but which may be in future.  Currently, the executables
remain in \texttt{src}.

\subsection{With VS 2015}
Solution files are stored in sub-directories of  \texttt{src}.
E.g. in  \texttt{src/corpusGenerator}

Properties are only set for the \texttt{Release}, \texttt{x64}
combination.  Changes to defaults for this combination include:

\begin{itemize}
\item Project/General:  Use multi-byte characters
  \item C/C++/Preprocessor: Define symbols \texttt{WIN64} and
    \texttt{\_CRT\_SECURE\_NO\_WARNINGS}
  \item C/C++/Precompiled headers: Not using precompiled headers
  \item C/C++/Advanced:  Compile as C
\end{itemize}

Open the solution file with VS 2015 and click Build/Rebuild.

The executable will be found in
e.g. \texttt{src/corpusGenerator/x64/Releases/corpusGenerator.exe}

\subsection{Building PDF documents}

\begin{verbatim}
cd doc
rm *.pdf
make
\end{verbatim}


\section{Notes on coding}

Project \projectName~has been developed in the course of scientific
research rather than engineered to commercial standards.  In many
cases, but not always, there is quite extensive internal documentation
at the head of modules or critical functions.
Unfortunately, there is inconsistency in variable
naming. The process of moving from the use\_of\_underscores to
``camelCase'' is regrettably incomplete.

Be warned that the principal author is highly enamoured of the
\texttt{unless} clause in perl.  Most scripts contain many examples
of:
\begin{verbatim}
die "message"
   unless <condition>;
\end{verbatim}

The C programs make very extensive use of ``dahash'' hash tables,
defined in \texttt{src/utils/dahash.[ch]}.  These hash tables avoid
the use of buckets.  Their size is constrained to be a power of two
and collisions are handled by relatively prime rehash.  Once a table
reaches a specified percentage of capacity, the table size is doubled
behind the scenes.  The hash function is the Fowler Noll Vo one.

Also defined is a mechanism for dynamic arrays --
\texttt{src/utils/dynamicArrays.[ch]}.  Accessing an array element
beyond allocated storage, causes the array to grow using a
configurable strategy.

C programs make extensive use of memory mapping through wrappers in
\texttt{src/utils/general.[ch]} which use either the
Windows style \texttt{CreateFileMapping()/MapViewOfFile()} (used if the
WIN64 symbol is defined) or the Unix interface \texttt{mmap()}.

All programs assume the use of UTF-8.  We recommend the use of the
iconv library \url{https://www.gnu.org/software/libiconv/} for
converting other character sets into UTF-8.  Functions for dealing
with UTF-8 are in \texttt{src/characterSetHandling/unicode.[ch]}.
Also in that directory is a perl script for updating Unicode tables
from the database at \url{http://unicode.org/}.


\section{Layout of Project Directories}

The project root directory contains the following files and directories:

\begin{description}
  \item[\texttt{doc}] - Contains the LaTeX source and PDFs of project documentation.
  \item[\texttt{Experiments}] - This is the directory under which all
    the output produced by scripts and executables will be stored.  If
    it doesn't exist the main perl scripts will create it.
\item[\texttt{LICENCE.TXT}] - See beginning of this document.
  \item[\texttt{ProjectGutenberg}] - Text versions of 49 books used to
    illustrate and validate the executables in this project.
  \item[\texttt{src}] - All the perl and C sources.  Currently
    executables are built in or under this directory.
\end{description}

\section{Desirable future developments}

\begin{enumerate}
  \item Code tidyup - variable names, internal documentation, module
    structure.  [Boring but useful]
    \item Additional document formats - or maybe converters from other
      formats into STARC format?  [Boring but useful]
      \item Piecewise growth models - Currently scaling up from a
        small corpus assumes the use of only a simple linear model for
        word frequency distribution.  [A challenge and very useful
          contribution.]
        \item Extending the supported term dependency models to include
          term repetition within a document, and term co-occurrence as
          well as n-grams. [A challenge and very useful
          contribution.]
        \item Efficiency monitoring and improvement, particularly
          reducing memory demands without exploding runtime.  [Could
            increase scope of application.]
          \item Reduce memory demands of \verb|corpusGenerator.exe| by
            using more memory-efficient representations of Markov
            transition matrices.  [Could
            increase scope of application.]
        \end{enumerate}

\section{Using \projectName}

Please see the companion User Overview document.


\end{document}
